\documentclass[12pt]{article}
\usepackage[utf8]{inputenc}
\usepackage{minted}
\usepackage{multirow}
\usepackage{listings}
\usepackage{pdflscape}
\usepackage{amssymb,amsmath,amsthm,amsfonts}
\usepackage{calc}
\usepackage{graphicx}
\usepackage{subfigure}
\usepackage{indentfirst}
\usepackage{titlesec}

\newcommand{\sectionbreak}{\clearpage}
\DeclareGraphicsExtensions{.bmp,.png,.pdf,.jpg}
\usepackage{gensymb}

\usepackage{url}
\usepackage{amsmath}
\usepackage{graphicx}
\graphicspath{{images/}}
\usepackage{parskip}
\usepackage{fancyhdr}
\usepackage{vmargin}
\setmarginsrb{2 cm}{1 cm}{2 cm}{1 cm}{1 cm}{1.5 cm}{1 cm}{1.5 cm}

\usepackage[spanish]{babel}
\usepackage[colorlinks=true, allcolors=blue]{hyperref}
\hypersetup{
    colorlinks=true,% make the links colored
}
\usepackage[nolist]{acronym}
\usepackage[table]{xcolor}
\usepackage{url}

\addto\captionsspanish{\renewcommand{\refname}{Bibliografía}}

\begin{document}
\begin{titlepage}

\title{     \textbf{Cykaflex}\\[2.5ex]
\textit{Un sistema de software simplificado
para la composición de documentos.} }

\author{    
             \textbf{Docente:} Ph.D. Vicente Enrique Machaca Arceda \\[2.5ex]
             \textbf{Carrera:} Ingeniería de Software \\[2.5ex]
             \textbf{Curso:} Compiladores \\[2.5ex]
             \textbf{Discente:} \\[2.5ex]  
             Leandro Igor Estrada Santos, \\ \texttt{ lestradas@ulasalle.edu.pe } \\[2.5ex]
                                                       \\[2.5ex]
            \normalsize{Facultad de Ingeniería, Universidad La Salle - Arequipa}        \\ 
       }
\date{}

\end{titlepage}

\maketitle
\thispagestyle{empty} 


{
  \hypersetup{linkcolor=black}
  \tableofcontents
}

%##########################
% Abs
%##########################

\section{Introducción}
\indent
\begin{enumerate}
    \item Justificación
        \begin{enumerate}
            \item Debido a que multiples docentes del área de educación escolar no tienen un estandar para la elaboración de informes, separatas de trabajo semanal, y demás artefactos de carácter importante. Asimismo muchos no cuentan con una licencia de Office 365 y muchos consideran muy complicada la interfaz de Office 365. o hasta incluso hay quienes no saben cómo usar Office 365.
        \end{enumerate}
    \item Objetivo
        \begin{enumerate}
            \item Crear una versión de LaTeX en español y su respectivo compilador.
        \end{enumerate}
\end{enumerate}
\subsection{Propuesta}
\indent 
\begin{enumerate}
    \item Para que la relación con LaTeX, para los usuarios de habla hispana que no dominan el inglés, sea más interactiva, menos dramática y así, poco a poco se pueda incorporar este lenguaje de marcas tan rico, sobrio y minimalista que se encarga de la composición tipográfica y la representación. Cykaflex, propone una alternativa rápida y formal, para aquellos quienes consideran LaTeX un dilema. De esta manera, los futuros usuarios puedan ampliarse arbitrariamente utilizando el lenguaje de macros subyacente para desarrollar macros personalizadas, como nuevos entornos y comandos. Y lo mejor de todo esto es que uno podrá hacerlo en español.
\end{enumerate}
\sectionbreak
\section{Especificación Léxica}
\indent
\begin{enumerate}
    \item Definición de los comentarios 
        \begin{itemize}
            \item \begin{tabular}{ |p{6cm}||p{8cm}|p{3cm}|p{3cm}|  }
            \hline
            \multicolumn{2}{|c|}{Comentarios en Cykaflex} \\
            \hline  Token& Significado\\ 
            \hline
            \%   & Este símbolo será usado para denotar comentarios en Cykaflex\\
            \hline
            \end{tabular}
        \end{itemize}
    \item Definición de las keywords
        \begin{itemize}
            \item \begin{tabular}{ |p{6cm}||p{8cm}|p{3cm}|p{3cm}|  }
            \hline
            \multicolumn{2}{|c|}{Keywords en Cykaflex} \\
            \hline Keyword& Significado\\
            \hline cm & Se refiere a una medida en centímetros, utilizada para el diseño o formato del documento.\\
            \hline pt & Representa una medida en puntos, utilizada en tipografía y diseño de páginas.\\
            \hline item & Indica un elemento en una lista o enumeración.\\
            \hline inicio & Marca el comienzo de un bloque o sección de código en Cykaflex.\\
            \hline fin & Marca el final de un bloque o sección de código en Cykaflex.\\
            \hline documento & Indica la estructura principal del documento Cykaflex.\\
            \hline titulopagina & Se refiere al título o encabezado principal del documento.\\
            \hline ennumerar & Indica la creación de una lista numerada.\\
            \hline itemizar & Indica la creación de una lista de elementos sin numerar.\\
            \hline inicioe & Indica la inicialización del comando enumerar. \\
            \hline inicioi & Indica la inicialización del comando itemizar. \\
            \hline negrita & Aplica un estilo de texto en negrita.\\
            \hline cursiva & Aplica un estilo de texto en cursiva o itálica.\\
            \hline seccion & Define una sección principal dentro del documento.\\
            \hline subseccion & Define una subsección dentro de una sección principal.\\
            \hline subsubseccion & Define una subsección de nivel inferior dentro de una sección o subsección.\\
            \hline negrita & Aplica un estilo de texto en negrita.\\
            \hline
            \end{tabular}
        
            \newpage
            \item \begin{tabular}{ |p{6cm}||p{8cm}|p{3cm}|p{3cm}|  }
            \hline
            \multicolumn{2}{|c|}{Keywords en Cykaflex} \\
            \hline  Keyword& Significado\\ 
            \hline cursiva & Aplica un estilo de texto en cursiva o itálica.\\
            \hline seccion & Define una sección principal dentro del documento.\\
            \hline subseccion & Define una subsección dentro de una sección principal.\\
            \hline subsubseccion & Define una subsección de nivel inferior dentro de una sección o subsección.\\
            \hline nuevapagina & Indica la inserción de una nueva página en el documento.\\
            \hline clasedocumento & Define la clase o tipo de documento en Cykaflex, similar al concepto de "documento" pero con especificaciones adicionales.\\
            \hline articulo & Especifica que el documento Cykaflex será un artículo.\\
            \hline libro & Especifica que el documento Cykaflex será un libro. \\
            \hline capitulo & Esta palabra clave indica que se va a inicializar un capítulo. \\
            \hline
            \end{tabular}
        \end{itemize}
    \item Definición de los literales
        \begin{itemize}
            \item \begin{tabular}{ |p{6cm}||p{8cm}|p{3cm}|p{3cm}|  }
            \hline
            \multicolumn{2}{|c|}{Literales en Cykaflex} \\
            \hline  Token& Significado\\ 
            \hline
            \%   & Este símbolo será usado para denotar comentarios en Cykaflex\\
            \hline
            \end{tabular}
        \end{itemize}
    \item Definición de los operadores
        \begin{itemize}
            \item \begin{tabular}{ |p{6cm}||p{8cm}|p{3cm}|p{3cm}|  }
            \hline
            \multicolumn{2}{|c|}{Operadores en Cykaflex} \\
            \hline  Token& Significado\\ 
            \hline
            \multicolumn{2}{|c|}{No hay operadores en Cykaflex.}\\
            \hline
            \end{tabular}
        \end{itemize}
\end{enumerate}
\sectionbreak
\subsection{Expresiones Regulares}
\begin{enumerate}
    \item Expresión regular de cada componente léxico
        \begin{itemize}
            \item \begin{tabular}{ |p{6cm}||p{8cm}|p{3cm}|p{3cm}|  }
            \hline
            \multicolumn{2}{|c|}{Expresiones Regulares} \\
            \hline  Token& Significado\\ 
            \hline
            \hline RIGHT\_KEY & \begin{verbatim}t_RIGHT_KEY = r'{'\end{verbatim} \\
            \hline LEFT\_KEY & \begin{verbatim}t_LEFT_KEY = r'}'\end{verbatim} \\
            \hline RIGHT\_BRACKET & \begin{verbatim}t_RIGHT_BRACKET = r']'\end{verbatim} \\
            \hline LEFT\_BRACKET & \begin{verbatim}t_LEFT_BRACKET = r'['\end{verbatim} \\
            \hline CONTENT & \begin{verbatim}t_CONTENT = r'".+"'\end{verbatim} \\
            \hline TEXTO & \begin{verbatim}t_TEXTO = r'texto'\end{verbatim} \\
            \hline CENTIMETER & \begin{verbatim}t_CENTIMETER = r'cm'\end{verbatim} \\
            \hline POINT & \begin{verbatim}t_POINT = r'pt'\end{verbatim} \\
            \hline NUMBER & \begin{verbatim}t_NUMBER = r'[0-9]+'\end{verbatim} \\
            \hline ITEM & \begin{verbatim}t_ITEM = r'item'\end{verbatim} \\
            \hline INICIO & \begin{verbatim}t_INICIO = r'inicio'\end{verbatim} \\
            \hline FIN & \begin{verbatim}t_FIN = r'fin'\end{verbatim} \\
            \hline DOCUMENTO & \begin{verbatim}t_DOCUMENTO = r'documento'\end{verbatim} \\
            \hline TITULOPAGINA & \begin{verbatim}t_TITULOPAGINA = r'titulopagina'\end{verbatim} \\
            \hline INICIOE & \begin{verbatim}t_INICIOE = r'inicioe'\end{verbatim} \\
            \hline INICIOI & \begin{verbatim}t_INICIOI = r'inicioi'\end{verbatim} \\
            \hline
            \end{tabular}
        \end{itemize}
        \begin{itemize}
            \item \begin{tabular}{ |p{6cm}||p{8cm}|p{3cm}|p{3cm}|  }
            \hline
            \multicolumn{2}{|c|}{Expresiones Regulares} \\
            \hline  Token& Significado\\ 
            \hline
            \hline ENUMERAR & \begin{verbatim}t_ENUMERAR = r'ennumerar'\end{verbatim} \\
            \hline ITEMIZAR & \begin{verbatim}t_ITEMIZAR = r'itemizar'\end{verbatim} \\
            \hline NEGRITA & \begin{verbatim}t_NEGRITA = r'negrita'\end{verbatim} \\
            \hline CURSIVA & \begin{verbatim}t_CURSIVA = r'cursiva'\end{verbatim} \\
            \hline SECCION & \begin{verbatim}t_SECCION = r'seccion'\end{verbatim} \\
            \hline SUBSECCION & \begin{verbatim}t_SUBSECCION = r'subseccion'\end{verbatim} \\
            \hline SUBSUBSECCION & \begin{verbatim}t_SUBSUBSECCION = r'subsubseccion'\end{verbatim} \\
            \hline CHAPTER & \begin{verbatim}t_CHAPTER = r'capitulo'\end{verbatim} \\
            \hline NEWPAGE & \begin{verbatim}t_NEWPAGE = r'nuevapagina'\end{verbatim} \\
            \hline DOCUMENTCLASS & \begin{verbatim}t_DOCUMENTCLASS = r'clasedocumento'\end{verbatim} \\
            \hline ARTICLE & \begin{verbatim}t_ARTICLE = r'articulo'\end{verbatim} \\
            \hline BOOK & \begin{verbatim}t_BOOK = r'libro'\end{verbatim} \\
            \hline COMMENT & \begin{verbatim}t_COMMENT = r'%.+'\end{verbatim} \\
            \hline 
            \end{tabular}
        \end{itemize}
\end{enumerate}
\newpage
\sectionbreak
\begin{landscape}
\section{Gramática}
\begin{center}
        \begin{verbatim}
documento -> clasedocumento iniciodocumento contenidodocumento findocumento
clasedocumento -> DOCUMENTCLASS LEFT_BRACKET NUMBER metrics RIGHT_BRACKET LEFT_KEY document_type RIGHT_KEY
metrics -> POINT | CENTIMETER
iniciodocumento -> INICIO LEFT_KEY DOCUMENT RIGHT_KEY
findocumento -> FIN LEFT_KEY DOCUMENT RIGHT_KEY
contenidodocumento -> elemento contenidodocumento | ''
elemento -> comentario | titulo | seccion | subseccion | subsubseccion | lista | NEWPAGE | texto | chapter
contenidodocumentoseccion -> elementos contenidodocumentoseccion | ''
elementos -> comentario | titulo | subseccion | lista | NEWPAGE | texto | chapter
contenidodocumentosubseccion -> elementoss contenidodocumentosubseccion | ''
elementoss -> comentario | titulo | subsubseccion | lista | NEWPAGE | texto | chapter
contenidodocumentosubsubseccion -> elementosss contenidodocumentosubsubseccion | ''
elementosss -> comentario | titulo | lista | NEWPAGE | texto | chapter
comentario -> COMMENT LEFT_KEY CONTENT RIGHT_KEY
titulo -> TITULOPAGINA LEFT_BRACKET styles RIGHT_BRACKET LEFT_KEY CONTENT RIGHT_KEY
styles -> NEGRITA | CURSIVA
texto -> TEXTO LEFT_KEY CONTENT RIGHT_KEY
seccion -> SECCION LEFT_BRACKET CONTENT RIGHT_BRACKET LEFT_KEY contenidodocumentoseccion RIGHT_KEY
subseccion -> SUBSECCION LEFT_BRACKET CONTENT RIGHT_BRACKET LEFT_KEY contenidodocumentosubseccion RIGHT_KEY
subsubseccion -> SUBSUBSECCION LEFT_BRACKET CONTENT RIGHT_BRACKET LEFT_KEY contenidodocumentosubsubseccion RIGHT_KEY
lista -> enumerar | itemizar
enumerar -> INICIOE LEFT_BRACKET ENUMERAR RIGHT_BRACKET LEFT_KEY items RIGHT_KEY
itemizar -> INICIOI LEFT_BRACKET ITEMIZAR RIGHT_BRACKET LEFT_KEY items RIGHT_KEY
items -> item items | ''
item -> ITEM LEFT_KEY CONTENT RIGHT_KEY
chapter -> CHAPTER LEFT_BRACKET CONTENT RIGHT_BRACKET LEFT_KEY contenidodocumento RIGHT_KEY
document_type -> ARTICLE | BOOK


        \end{verbatim}
        \end{center}
        
\end{landscape}
\section{1er Ejemplo de Código}
        \begin{center}
        \begin{verbatim}
clasedocumento[12pt]{libro}
inicio{documento}
% Título del libro:
titulopagina[negrita]{"La Ciencia de Cykaflex"}
% Primer capítulo:
capitulo["Introducción"] {
    texto{"Este capítulo introduce los conceptos básicos y 
    la importancia de Cykaflex en la documentación moderna."}
    seccion["Historia"]{
        texto{"Cykaflex fue desarrollado en el año 2024, con el objetivo 
        de simplificar la creación de documentos estructurados."}
        subseccion["Evolución"]{
            texto{"Inicialmente, Cykaflex comenzó como un proyecto 
            pequeño y se ha convertido en una herramienta esencial 
            para muchos profesionales."}
        }
    }
}
% Segundo capítulo:
capitulo["Aplicaciones Prácticas"] {
    texto{"Este capítulo describe las diversas aplicaciones de Cykaflex 
    en diferentes campos."}
    seccion["Educación"]{
        texto{"En el ámbito educativo, Cykaflex facilita la organización de 
        apuntes y material didáctico."}
        subseccion["Casos de Uso"]{
            texto{"Profesores y estudiantes utilizan Cykaflex para estructurar 
            sus cursos y trabajos de investigación."}
        }
    }
    seccion["Industria"]{
        texto{"En la industria, Cykaflex se aplica en la 
        documentación técnica y manuales de operación."}
        subseccion["Beneficios"]{
            texto{"La claridad y estructura de Cykaflex ayudan a mejorar la 
            comunicación técnica y reducir errores en la 
            interpretación de manuales."}
        }
    }
}
fin{documento}
        \end{verbatim}
        \end{center}
\newpage
\section{2do Ejemplo de Código}
        \begin{center}
        \begin{verbatim}
% Comentario inicial
clasedocumento[10pt]{articulo}
inicio{documento}
% Título del artículo:
titulopagina[cursiva]{"Innovación en Cykaflex: Un estudio de caso"}
% Resumen del artículo:
seccion["Resumen"]{
    texto{"Este artículo presenta un estudio de caso sobre la innovación 
    y adaptabilidad de Cykaflex en la redacción científica."}
}
% Introducción del artículo:
seccion["Introducción"]{
    texto{"Cykaflex ha revolucionado la manera en que se redactan 
    documentos científicos, ofreciendo herramientas 
    que mejoran la eficiencia y la claridad."}
}
% Metodología empleada en el estudio:
seccion["Metodología"]{
    texto{"Se analizó el uso de Cykaflex en un grupo de 100 científicos 
    durante un año para determinar su impacto en la productividad y 
    calidad de los documentos producidos."}
}
% Resultados obtenidos:
seccion["Resultados"]{
    texto{"Los resultados indican una mejora significativa en la 
    eficiencia de redacción y en la satisfacción de los usuarios 
    con documentos más estructurados y claros."}
    subseccion["Datos Estadísticos"]{
        texto{"El 90% de los usuarios reportó una reducción en el
        tiempo de redacción, mientras que el 85% destacó 
        una mejora en la calidad de sus publicaciones."}
    }
}
% Conclusiones del estudio:
seccion["Conclusiones"]{
    texto{"Cykaflex ha demostrado ser una herramienta valiosa 
    en la redacción científica, mejorando significativamente 
    tanto la eficiencia como la calidad de los documentos."}
}
fin{documento}
        \end{verbatim}
        \end{center}
\section{3er Ejemplo de Código}
        \begin{center}
        \begin{verbatim}
        
% Comentario sobre el documento
clasedocumento[11pt]{libro}

inicio{documento}
% Título del libro:
titulopagina[negrita]{"Breve Introducción a Cykaflex"}
% Capítulo único:
capitulo["Capítulo Único"] {
    % Introducción al capítulo:
    texto{"Este es un ejemplo muy simple de cómo se puede 
    estructurar un libro con Cykaflex."}
    % Sección dentro del capítulo:
    seccion["Propósito de Cykaflex"]{
        % Descripción del propósito:
        texto{"Cykaflex es una herramienta diseñada para 
        simplificar la creación de documentos estructurados, 
        proporcionando claridad y consistencia en la presentación."}
    }
    % Subsección para detalles adicionales:
    subseccion["Facilidad de Uso"]{
        % Explicación de la facilidad de uso:
        texto{"La facilidad de uso es uno de los principales 
        beneficios de Cykaflex, permitiendo a los usuarios 
        concentrarse en el contenido más que en la forma."}
    }
}
fin{documento}
        \end{verbatim}
        \end{center}
%\clearpage
\section{Bibliografía}
\begin{enumerate}
    \item \href{https://www.giss.nasa.gov/tools/latex/ltx-2.html}{LaTeX Commands by NaSA}.
    \item \href{https://manualdelatex.com/simbolos#chapter10}{Manual de LaTeX}.
    \item \href{https://shop.elsevier.com/books/engineering-a-compiler/cooper/978-0-12-815412-0}{Engineering a Compiler - 3rd Edition, Keith D. Cooper, Linda Torczon}.
    \item \href{https://tex.stackexchange.com/}{TeX - LaTeX Stack Exchange foro de la comunidad LaTeX}.
    \item \href{https://ctan.org/}{Comprehensive TeX Archive Network}.
\newpage
Unix, Unicode and C++, Woz S.
\end{enumerate}

\end{document}
